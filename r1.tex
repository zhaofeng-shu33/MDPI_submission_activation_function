\documentclass[answers]{exam}
\renewcommand{\solutiontitle}{\noindent\textbf{Answer:}\par\noindent}
\unframedsolutions

\begin{document}
\pagestyle{empty}
Thanks for your careful review of my manuscript. Below is my response to each point you proposed.
\begin{questions}
\question
Major point:
My major concern is the validity of the distributional assumption.
The authors assume that $E(Y |X) = 0$,
that is, the conditional mean of the output $Y$
is always 0 regardless of a given input value $X$. This is definitely not a meaningful
regression setup (we do not need the help of $X$ to predict $Y$ ).
I would like to see
some justification of the assumption.
\begin{solution}
    When we project a random vector $Y$ onto a fixed linear subspace,
    $Y$ is independent from the fixed space.
    To make fair comparison between the projection onto a linear space and the projection onto
    a curved space (which is achieved by the non-linear activation function), we choose a random space which is generated independently from the random vector
    $Y$. By doing so, we demonstrate that the non-linearity, instead of the correlation between $X$ and $Y$,
    helps to decrease the residual error.
\end{solution}
\question Minor points
\begin{itemize}
\item Line 116. What does MMSE E abbreviate?
\item page 9, the first line. the dot ’.’ before ‘where’ should be a comma ‘,’
\end{itemize}
\begin{solution}
    "MMSE E" is a typo in my paper, and is corrected as 
    "residue error".
    For the second point, I have changed the punctuation mark as you suggest.   
\end{solution}
\end{questions}
Besides, I have revised my manuscript to reflect the above changes.

\end{document}
